\documentclass{beamer}
\usepackage{siunitx}

\title{Introduction to Linux}
\subtitle{History, Background, and Applications}
\author{Siddharth Patil}
\date{25 November 2015}

\usecolortheme{wolverine}

\begin{document}
\frame{\titlepage}

\begin{frame}
  \frametitle{What is Linux?}
  \begin{itemize}
  \item Properly known as GNU/Linux \begin{itemize}
    \item Linux specifically refers to kernel, while GNU refers to userspace tools
    \end{itemize}
  \item Unix-like computer operating system
  \item Free/Open Source Software (FOSS) \begin{itemize}
    \item Free to modify source code and redistribute, even commercially
    \item Open development process
    \end{itemize}
  \end{itemize}
\end{frame}

\begin{frame}
  \frametitle{What is Linux?}
  \framesubtitle{Prehistory}
  \begin{itemize}
  \item Unix operating system created at AT\&T Bell Laboratories in 1969--1971, by Ken Thompson and Dennis Ritchie \begin{itemize}
    \item Sold to academic and commercial institutions
    \item Was too expensive for private users
    \end{itemize} \pause
  \item Berkeley Software Distribution (BSD) created at UC Berkeley in 1977, based on AT\&T Unix code \begin{itemize}
    \item Codebase eventually utilized by Apple's Mac OS X line of operating systems
    \item Usage of AT\&T Unix code led to litigation that slowed BSD's adoption
    \end{itemize} \pause
  \item GNU Project started in 1983, by Richard Stallman \begin{itemize}
    \item Intended to be a free Unix-like operating system
    \item Original kernel (GNU Hurd) adopted a design that proved too difficult to develop quickly
    \end{itemize} \pause
  \item MINIX kernel started in 1987, by Andrew S. Tanenbaum \begin{itemize}
    \item Intended for educational use
    \item Original license too restrictive
    \end{itemize}
  \end{itemize}
\end{frame}

\begin{frame}
  \frametitle{What is Linux?}
  \framesubtitle{History}
  \begin{itemize}
  \item Kernel project started in 1991, by Linus Torvalds
  \item Project quickly turns into Linux kernel
    \begin{itemize}
    \item Code relicensed under GNU General Public License (GPL)
    \item Relicensing removes restriction on commercial redistribution
    \end{itemize}
  \item Developers all around the world start contributing to Linux codebase
  \item Linux kernel integrated with GNU Project tools
  \end{itemize}
\end{frame}

\begin{frame}
\frametitle{What is Linux?}
\framesubtitle{Today}
  \begin{itemize}
  \item Over \SI{80}{\percent} of Linux developers are paid
  \item Linux kernel at stable version 4.3
  \item Majority of smartphones, supercomputers, and consumer routers use Linux
  \end{itemize}
\end{frame}

\begin{frame}
  \frametitle{Where is Linux used?}
  \framesubtitle{Consumer-grade applications}
  \begin{itemize}
  \item Personal computers \begin{itemize}
    \item Known as Linux distributions
    \item Each distributions designed with specific features for specific uses and markets
    \item Ubuntu, Fedora, Arch, Gentoo, etc.
    \end{itemize} \pause
  \item Smartphones and Tablets \begin{itemize}
    \item Google's Android platform uses the Linux kernel
    \end{itemize} \pause
  \item Embedded systems \begin{itemize}
    \item TVs
    \item Consumer routers
    \item In-car entertainment systems
    \end{itemize}
  \end{itemize}
\end{frame}

\begin{frame}
  \frametitle{Where is Linux used?}
  \framesubtitle{Large-scale applications}
  \begin{itemize}
  \item Web servers \begin{itemize}
    \item LAMP solution stack commonly used
    \item Linux, Apache, MySQL, Python/Perl/PHP
    \end{itemize} \pause
  \item Supercomputers \begin{itemize}
    \item Majority (\SI{98.8}{\percent}) of world's 500 fastest supercomputers (TOP500) run Linux distributions
    \end{itemize} \pause
  \item Mainframes \begin{itemize}
    \item IBM embraced Linux in 2000
    \item Linux now powers majority of IBM's current offerings
    \end{itemize}
  \end{itemize}
\end{frame}

\begin{frame}
  \frametitle{Advantages of Linux}
  \begin{itemize}
  \item Unix-based design is an efficient approach
  \item Highly modular, making target optimization easier
  \item FOSS nature makes it relatively easy to obtain, use, and develop/modify
    \begin{itemize}
    \item Has been ported to many architectures as a result \pause
      \begin{itemize}
      \item x86, x86-64 --- Personal computers, servers, workstations \pause
      \item SPARC --- Sun Microsystems servers, workstations \pause
      \item ARM --- Smartphones, tablets \pause
      \item MIPS --- Lower-cost smartphones and tablets, Sony PlayStation 2, Loongson and Lemote personal computers \pause
      \item POWER --- IBM servers \pause
      \item PowerPC --- pre-2006 Apple Macintosh personal computers, Sony PlayStation 3
      \end{itemize}
    \end{itemize}
  \end{itemize}
\end{frame}

\begin{frame}
  \frametitle{Conclusion}
  \begin{itemize}
  \item Linux is a very versatile operating system
  \item Its free/open source nature is one of its biggest advantages
  \item Expect to see Linux adoption grow long-term
  \end{itemize}
\end{frame}

\end{document}